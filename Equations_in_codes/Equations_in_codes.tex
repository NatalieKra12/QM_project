\documentclass[11pt,a4paper,oneside]{article}
\usepackage{geometry}
\geometry{left=25mm, right=25mm, top=20mm, bottom=25mm}
% komendy umożliwiające obsługę języka polskiego i znaków matematycznych 
\usepackage[utf8]{inputenc}
\usepackage{polski}
\usepackage{amsmath}
\usepackage{amsfonts}
% komendy umożliwiające wrzucanie obrazków i hiperlinków
\usepackage{graphicx}
\usepackage{hyperref}
\usepackage{float}
\usepackage{graphicx}
\usepackage{epsfig}
\usepackage{enumitem}
\usepackage{pythontex}
\begin{document}
\section{Równanie Schrödingera dla nieskończonej dwuwymiarowej studni potencjału}
	Mamy studnię potencjału o wymiarach $a=1$, $b=1$. Liczby kwantowe $n_{x}$, $n_{y}$.
	\newline
	
	\noindent Równanie Schrödingera jest dane wzorem
	\begin{equation}
		i\hbar \frac{\delta \Psi}{\delta t} = -\frac{\hbar^{2}}{2m}\frac{\delta ^{2} \Psi}{\delta x ^{2}}+V \Psi.
	\end{equation}
	Potencjał rozważanej studni wynosi
	\begin{equation}
			V =
			\begin{cases}
				0 & x\in [0, a], y\in [0, b]\\
				\infty & wpp,
			\end{cases} 
	\end{equation}
	Z metody rozdzielania zmiennych i warunku unormowania
	\begin{equation}
		X(x) = \sqrt{\frac{2}{a}}\sin \left(\frac{n_{x}\pi x}{a}\right ),
	\end{equation}
	oraz:
	\begin{equation}
		Y(y) = \sqrt{\frac{2}{b}}\sin \left(\frac{n_{y}\pi y}{b}\right ).
	\end{equation}
	Stąd rozwiązanie równania Schrödingera dla rozważanej studni
	\begin{equation}
		\Psi (x, y, t) = \sqrt{\frac{2}{a}}\sin \left(\frac{n_{x}\pi x}{a}\right )\sqrt{\frac{2}{b}}\sin \left(\frac{n_{y}\pi y}{b}\right )e^{\frac{-iEt}{\hbar ^{2}}}.
	\end{equation}
\section{Poziomy energetyczne}
	Energia jest dana wzorem
	\begin{equation}
		E = E_{x}+E_{y},
	\end{equation}
	\begin{equation}
		E_{x} = \frac{n_{x} ^{2} \pi ^{2} \hbar ^{2}}{a^{2} 2 m}
	\end{equation}
	oraz
	\begin{equation}
		E_{y} = \frac{n_{y} ^{2} \pi ^{2} \hbar ^{2}}{a^{2} 2 m}.
	\end{equation}
	Stąd poziomy energetyczne dane są wzorem
	\begin{equation}
		E_{n_{x}, n_{y}} = \frac{\pi ^{2} \hbar ^{2}}{2m} (n_{x}^2+n_{y}^{2}).
	\end{equation}
\section{Stany nieustalone nieskończonej dwuwymiarowej studni potencjału}
	Stany nieustalone są dane wzorem
	\begin{equation}
		\Psi (x, t) = \sum_{n=1}^{N} c_{n}\Psi_{n}(x, t)e^{\frac{-iE_{x}t}{\hbar ^{2}}}
	\end{equation}
	oraz
	\begin{equation}
		\Psi (y, t) = \sum_{m=1}^{N} c_{m}\Psi_{m} e^{\frac{-iE_{y}t}{\hbar ^{2}}}.
	\end{equation}
	Więc:
	\begin{equation}
		\Psi(x, y, t) = \sum_{n=1}^{N} c_{n}\Psi_{n}(x, t)\sum_{m=1}^{N} c_{m}\Psi_{m} e^{\frac{-iEt}{\hbar ^{2}}},
	\end{equation}
	Energia jest dana wzorem
	\begin{equation}
		E = E_{x}+E_{y}
	\end{equation}
	Współczynniki w pzypadku rozważanej studni są unormowane:
	\begin{equation}
		\sum_{n=1}^{N}|c_{n}|^{2} = 1.
	\end{equation}
\section{Paczka gaussowska w dwuwymmiarowej nieskończonej studni potencjału}
	Rozwiazanie ogólne równania Schrödingera jest liniową kombinacją osobnych rozwiązań
	\begin{equation}
		\Psi(x, y, t) = \sum_{n=1}^{N} c_{n}\Psi_{n}(x, t)\sum_{m=1}^{N} c_{m}\Psi_{m} e^{\frac{-iEt}{\hbar ^{2}}}.
	\end{equation}
	Współczynniki wyznaczone są wzorem
	\begin{equation}
		c_{n} = \int_{0}^{L}\Psi _{n}(x)^{*}\Psi (x,0)dx,
	\end{equation}
	\begin{equation}
		c_{m} = \int_{0}^{L}\Psi _{m}(y)^{*}\Psi (y,0)dy.
	\end{equation}	
	Warunek początkowy funkcji falowej dla paczki gaussowskiej jest dany rozkładem normalnym
	\begin{equation}
		\Psi(x, 0) = Ae^{\frac{-(x-a_{0})^{2}}{2\sigma ^{2}}}e^{ik_{0}x}
	\end{equation}
	oraz
	\begin{equation}
		\Psi(y, 0) = Ae^{\frac{-(y-b_{0})^{2}}{2\sigma ^{2}}}e^{ik_{0}y}.
	\end{equation}	
\end{document}